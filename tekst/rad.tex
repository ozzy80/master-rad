% Format teze zasnovan je na paketu memoir
% http://tug.ctan.org/macros/latex/contrib/memoir/memman.pdf ili
% http://texdoc.net/texmf-dist/doc/latex/memoir/memman.pdf
% 
% Prilikom zadavanja klase memoir, navedenim opcijama se podešava 
% veličina slova (12pt) i jednostrano štampanje (oneside).
% Ove parametre možete menjati samo ako pravite nezvanične verzije
% mastera za privatnu upotrebu (na primer, u b5 varijanti ima smisla 
% smanjiti 
\documentclass[12pt,oneside]{memoir} 

% Paket koji definiše sve specifičnosti master rada Matematičkog fakulteta
\usepackage[latinica,biblatex]{matfmaster} 
\usepackage{listings}
\usepackage{verbatim}
%
% Podrazumevano pismo je ćirilica.
%   Ako koristite pdflatex, a ne xetex, sav latinički tekst na srpskom jeziku
%   treba biti okružen sa \lat{...} ili \begin{latinica}...\end{latinica}.
%
% Opicija [latinica]:
%   ako želite da pišete latiniciom, dodajte opciju "latinica" tj.
%   prethodni paket uključite pomoću: \usepackage[latinica]{matfmaster}.
%   Ako koristite pdflatex, a ne xetex, sav ćirilički tekst treba biti
%   okružen sa \cir{...} ili \begin{cirilica}...\end{cirilica}.
%
% Opcija [biblatex]:
%   ako želite da koristite reference na više jezika i umesto paketa
%   bibtex da koristite BibLaTeX/Biber, dodajte opciju "biblatex" tj.
%   prethodni paket uključite pomoću: \usepackage[biblatex]{matfmaster}
%
% Opcija [b5paper]:
%   ako želite da napravite verziju teze u manjem (b5) formatu, navedite
%   opciju "b5paper", tj. prethodni paket uključite pomoću: 
%   \usepackage[b5paper]{matfmaster}. Tada ima smisla razmisliti o promeni
%   veličine slova (izmenom opcije 12pt na 11pt u \documentclass{memoir}).
%
% Naravno, opcije je moguće kombinovati.
% Npr. \usepackage[b5paper,biblatex]{matfmaster}

% Pomoćni paket koji generiše nasumičan tekst u kojem se javljaju sva slova
% azbuke (nema potrebe koristiti ovo u pravim disertacijama)
\usepackage[latinica]{pangrami}

% Datoteka sa literaturom u BibTex tj. BibLaTeX/Biber formatu
\bib{literatura}

% Ime kandidata na srpskom jeziku (u odabranom pismu)
\autor{Ozren Demonja}
% Naslov teze na srpskom jeziku (u odabranom pismu)
\naslov{Realizacija P2P protokola za dostavu sinhronizovanog sadržaja}
% Godina u kojoj je teza predana komisiji
\godina{2018}
% Ime i afilijacija mentora (u odabranom pismu)
\mentor{dr Aleksandar \textsc{Kartelj}, docent\\ Univerzitet u Beogradu, Matematički fakultet}
% Ime i afilijacija prvog člana komisije (u odabranom pismu)
\komisijaA{dr Saša \textsc{Malkov}, profesor\\  Univerzitet u Beogradu, Matematički fakultet}
% Ime i afilijacija drugog člana komisije (u odabranom pismu)
\komisijaB{dr Miroslav \textsc{Marić}, profesor \\ Univerzitet u Beogradu, Matematički fakultet}
% Datum odbrane (odkomentarisati narednu liniju i upisati datum odbrane ako je poznat)
% \datumodbrane{}

% Apstrakt na srpskom jeziku (u odabranom pismu)
\apstr{%
}

% Ključne reči na srpskom jeziku (u odabranom pismu)
\kljucnereci{P2P mreža, deljenje video sadržaja, sinhronizovano deljenje sadržaja, P2P protokol, deljenje u realnom vremenu, decentralizovana razmena podataka}

\begin{document}
% ==============================================================================
% Uvodni deo teze
\frontmatter
% ==============================================================================
% Naslovna strana
\naslovna
% Strana sa podacima o mentoru i članovima komisije
\komisija
% Strana sa posvetom (u odabranom pismu)
\posveta{Mami i tati}
% Strana sa podacima o disertaciji na srpskom jeziku
\apstrakt
% Sadržaj teze
\tableofcontents*

% ==============================================================================
% Glavni deo teze
\mainmatter
% ==============================================================================

% ------------------------------------------------------------------------------
\chapter{Uvod}
% ------------------------------------------------------------------------------
\chapter{P2P mreža}
\label{chp:p2p-uvod}

Osnovna svrha interneta kao globalne mreže u današnje vreme je razmena digitalnog sadržaja. Arhitektura koja se koristi za podršku razmeni često je zasnovana na modelu klijent-server (eng. \textit{Client-Server architecture}). Model klijent-server deli zadatke između servera koji obezbeđuju resurse ili usluge i klijenata koji zahtevaju usluge \cite{DeBoever07}.

Kako svaki pristup sistemu od strane klijenta zahteva veću upotrebu resursa i bolje performanse, jasno se uočava da je server usko grlo sistema. 
Tokom 1990-ih godina internet se najvećim delom zasnivao na klijent-server modelu. Poslednjih desetak godina dolazi do velikog proboja novih tehnologija usled želje korisnika za interaktivnijim sadržajem, porastom brzine pristupa internetu (eng. \textit{broadband}), rasprostranjenosti pristupa, boljom stabilnosti veze i jačim računarima korisnika.

Navedene činjenice dovode do promene načina korišćenja interneta i pravljenja mreže ravnopravnih članova (eng. \textit{peer-to-peer (P2P) network}) u kojoj korisnici dele svoje resurse \cite{Tanenbaum}. Primena ovakvog pristupa obećavajuće je rešenje za mnoge oblasti.

U poglavlju \ref{P2P.1} opisani su osnovni koncepti mreže ravnopravnih članova. U poglavlju \ref{P2P.2} opisane su struktuirane i nestruktuirane mreže ravnopravnih članova. U poglavlju \ref{P2P.3} opisane su topologija stabla (eng. \textit{tree topology}), mrežna (eng. \textit{mesh topology}) i hibridna (eng. \textit{hybrid topology}) topologija. Način pripreme i slanja video sadržaja opisan je u poglavlju \ref{P2P.4}. U poglavlju \ref{P2P.5} opisani su osnovni problemi prilikom pravljenja mreže ravnopravnih članova.


\section{Osnovni koncepti}
\label{P2P.1}
Korišćenje aplikativnog sloja u formiranju pokrivača (eng. \textit{overlays}) za pristup internet servisima ima dugu istoriju. I pored duge tradicije ovakav pristup se do nedavno koristio samo za dizajn specifičnih protokola i interno povezivanje servisa infrastrukture, bez mogućnosti korišćenja između dva računara koja nisu susedi. Popularizacija ideje za formiranje pokrivača na aplikativnom sloju počinje 1999. godine sa Napster-om \cite{Aberer_2004} i nastavlja se kroz druge sisteme za razmenu digitalnog sadržaja kao što su Gnutella \cite{Gnutella}, FastTrack \cite{FastTrack}, KaZaa \cite{kazaa} i BitTorrent \cite{Bittorent}. 

Arhitektura ravnopravnih članova (eng. \textit{peer-to-peer, P2P}) definiše se kao distribuirani vid arhitekture u kojoj svaki parnjak (eng. \textit{peer}) može samostalno da obavlja poslove, ali omogućava drugim parnjacima da koriste njegove resurse kao što i sam može po potrebi koristiti resurse drugih parnjaka \cite{Shen:2009}. Drugim rečima, svaki parnjak komunicira sa ostalim parnjacima u mreži koristeći softver koji se ponaša i kao klijent i kao server. Na osnovu navedenih činjenica zaključuje se da arhitektura ravnopravnih članova omogućava veću autonomiju članova mreže koji sami definišu pravila o deljenju resursa. Glavna primena arhitekture ravnopravnih članova prisutna je u sistemima sa većom tolerancijom greške kod distribuirane obrade. Pored toga, ovaj vid arhitekture koristi se prilikom razmene fajlova, direktne komunikacije, obrade velikih količina podataka i softvera za zabavu. 


Prilikom deljenja resursa svaki parnjak doprinosi resursima mreže ravnopravnih članova. Idealno, deljenje resursa je proporcionalno broju parnjaka koji koriste mrežu ravnopravnih članova. Međutim, usled različitih problema u praksi navedeno svojstvo većinom nije ispunjeno. 

P2P sistem čini skup međusobno povezanih parnjaka, direktno ili preko drugih parnjaka formirajući povezani graf. Kako u grafu ne postoji centralna tačka kontrole i kolektivno se obavlja posao, P2P sistemi su decentralizovani. U nekim situacijama definiše se centralni server za poslove kao što su identifikacija, autentikacija i različite sigurnosne provere. U ovakvim uslovima, po pravilu server obavlja što je moguće manji deo posla. Pored toga, ponekad se u P2P sistemima iz različitih razloga uvode parnjaci zaduženi za dodatni deo posla.
Ovakvi parnjaci se nazivaju super parnjaci (eng. \textit{super peers}).


Učešće parnjaka u obavljanju posla u P2P sistemima određeno je lokalno. Iz ove činjenice proizilazi bolja organizacija P2P sistema vremenom usled korišćenja lokalnih znanja i operacija na svakom parnjaku. Ovim se takođe sprečava neplanirana dominacija nekog parnjaka sistemom.

Dva važna svojstva P2P sistema su samoskalabilnost i stabilnost. 
Samoskalabilnost, kao glavna prednost P2P sistema,
ispoljava se povećanjem ukupne snage sistema prilikom pristupanja parnjaka. Stabilnost mreže se obično definiše kao njena otpornost na veliku stopu promene strukture mreže usled čestog dolaska i odlaska parnjaka. U literaturi ove promene se nazivaju talasanja (eng. \textit{churn}). P2P sistem bi trebao biti stabilan do određenog nivoa talasanja, u kontekstu održavanja grafa konekcije i upućivanja (eng. \textit{route}) sadržaja deterministički sa prihvatljivim granicama broja skokova (eng. \textit{hop-count bounds}).


P2P sistem, kao tehnologija u zamahu, privlači veliku pažnju običnih i korporativnih korisnika. Takođe, pored pažnje korisnika, značajno interesovanje beleži se i u istraživačkim krugovima \cite{DeBoever07}. U poslednjih nekoliko godina, veliki deo istraživanja bavi se P2P deljenjem resursa i prenosom podataka. 

\section{Klasifikacija P2P mreža}
\label{P2P.2}

Usled velikog broja kriterijuma dizajniranja P2P mreža, ne postoji jedinstven način klasifikacije \cite{Shen:2009}. Na primer, sistemi za deljenje fajlova često se klasifikuju po generacijama. Prvu generaciju karakteriše hibridni dizajn. Glavnu karakteristiku ove generacije predstavlja kombinacija servera sa P2P upućivanjem sadržaja. Druga generacija je zasnovana na decentralizovanoj arhitekturi. Trećom generacijom smatraju se anonimni P2P sistemi kao što su Freenet \cite{Clarke_2001} i I2P \cite{I2P}. Ovakav način kategorizacije prema generacijama ima značajnih nedostataka. Glavni nedostatak ogleda se u nedefinisanim važnim dimenzijama problema i nejasnoći dodatnih mogućnosti sledeće generacije. Takođe, sistemi sve tri generacije su korišćeni u isto vreme što se ne može zaključiti na osnovu njihovih imena. Zbog navedenih nedostataka, u literaturi se češće sreće podela prema P2P pokrivaču. Prema ovoj podeli, razlikujemo struktuiranu i nestruktuiranu P2P mrežu  \cite{Prasanna_Ganesan_2004}.

U struktuiranim P2P mrežama topologija je precizno definisana. 
Takođe, neophodna je striktna kontrola protoka sadržaja. Sadržaj se ne sme nalaziti kod proizvoljnih parnjaka već mora biti na strogo definisanim lokacijama koje će omogućiti efikasniju pretragu.
Većina struktuiranih P2P mreža je bazirana na distribuiranim heš tabelama (eng. \textit{Distributed Hash Table (DHT)}). Kao primere struktuiranih P2P mreža navodimo Tapestry \cite{Zhao_2004}, Chord \cite{Stoica_2003}, Pastry \cite{Rowstron_2001} i Kademlia \cite{Maymounkov_2002}.

Ne struktuirani P2P sistem je sačinjen od parnjaka koji se pridružuju mreži sa nekim labavim skupom pravila, bez ikakvog predznanja o njenoj topologiji. Freenet, Gnutella, FastTrack, KaZaA i BitTorrent su primeri ovakvih sistema. Ovakve mreže se zbog svojih svojstava tipično koriste za udruživanje snage parnjaka nebi li se neki posao brže obavio ili razmenu fajlova kod kojih nije od presudnog značaja brzina pronalaska fajla. 

U nestruktuiranim P2P mrežama, pretraga je bazirana na prosleđivanju upita. Svaki parnjak upit prosleđuje komšijama, osim u slučaju kada parnjak može odgovoriti na traženi upit ili se brojač skokova upita smanji na 0 \cite{Shen:2009}. Sa ovakvim načinom prosleđivanja postoje različita pravila kontrole prosleđivanja upita. Način prosleđivanja može se kontrolisati tako što se upit neće proslediti svim komšijama nego samo određenim. Izbor kome će upit biti prosleđen se vrši prema informacijama koje parnjak čuva o komšijama. Te informacije predstavljene su istorijom ili indeksima sadržaja koji komšija može obezbediti. Takođe, često se prosleđivanje vrši slučajno bez ikakvog znanja. Naravno, ovo svojstvo smanjuje šansu da se pronađe sadržaj. 
Opravdanje nestruktuiranog pristupa zasniva se na promenljivosti broja članova mreže i ispravnosti prvog izbora komšije.

Nestruktuirane mreže ravnopravnih računara uglavnom su sastavljene od parnjaka koji su kućni PC računari. Oni su najčešće povezani sa mrežom slabim propusnim opsegom (eng. \textit{bandwidth}). Takođe, većinom je propusni opseg nesrazmeran u odnosu na dolazni i odlazni saobraćaj u korist dolaznog saobraćaja. Navedena činjenica blago narušava samoskalabilnost mreže. Postoje različiti načini da se spreči narušavanje samoskalabilnosti. Neki od njih su zasnovani na kažnjavanju parnjaka koji ne doprinose ili jako malo doprinose resursima sistema. 


\section{Topologija}
\label{P2P.3}

P2P sistem je sačinjen od dva podsistema koji se izvršavaju paralelno. To su konstrukcija i održavanje izabrane topologije kao i prosleđivanje bloka (eng. \textit{chunk}) podataka. Algoritmi za konstrukciju i prosleđivanje zajedno formiraju P2P protokol. U zavisnosti od odluka donešenih pri dizajnu sistema tj. od izabrane topologije i algoritma prosleđivanja može se napraviti P2P sistem sa različitim karakteristikama.

\subsection{Topologija drveta}
\label{P2P.3.1}

Među prvim predloženim topologijama P2P sistema za razmenu sadržaja predstavljena je topologija drveta. U okviru ove topologije, svi parnjaci su organizovani tako da čine strukturu drveta. Primer topologije drveta prikazan je na slici \ref{fig:topologija-drveta}.
 
\begin{figure}[!ht]
  \centering
  \includegraphics[width=1.05\textwidth]{slike/tree-topology.jpg}
  \caption{Primer topologije drveta}
  \label{fig:topologija-drveta}
\end{figure}
\par

U korenu drveta nalazi se parnjak koji deli sadržaj-izvor. Svaki parnjak u sistemu može imati tačno određeni broj dece koji je ili unapred definisan ili određen kapacitetom parnjaka. Deca su komšije koje preuzimaju sadržaj od roditelja. Konstrukcija drveta i relacija roditelj-dete može biti određena različitim faktorima kao što su ukupno vreme potrebno da podatak stigne od izvora do svih parnjaka u mreži, dostupan propusni opseg i fizička topologija same mreže. Podaci koji se šalju na ovakav način se najpre podele na manje delove koji se nazivaju paketi. Paket se zatim šalje od izvora do parnjaka koji se nalaze na prvom nivou drveta. Nakon toga parnjaci sa prvog nivoa drveta šalju podatke parnjacima na drugom nivou drveta i tako redom sve dok paket ne dođe do svih parnjaka u mreži \cite{DeBoever07}.

Iako jednostavna za konstrukciju, topologija drveta ima dosta nedostataka. Najveći nedostatak je neravnomerna raspodela tereta prosleđivanja koja se javlja kao posledica činjenice da veliki deo parnjaka čine listovi koji ne daju nikakav doprinos sistemu. Dodatno, parnjak koji nema odlazni (eng. \textit{upload}) kapacitet proporcionalan proizvodu broja dece i veličine bloka koji prenosi, konstantno će nagomilavati blokove podataka i usporavati ceo sistem. Ovakav parnjak treba da bude lociran jedino u listovima stabla ili veoma blizu listova. 
Pored navedenih osobina, važno je napomenuti da je ovakva struktura ranjiva na napuštanje parnjaka. Prilikom napuštanja parnjaka na višem nivou, njegova podstabla privremeno ostaju odsečena od ostatka mreže.

Da bi se povećala otpornost na napuštanja parnjaka i povećao kapacitet mreže koristi se rešenje bazirano na višestrukim stablima. U ovakvim sistemima parnjaci se nalaze u jednom ili više stabala. Svaki parnjak prima sadržaj od svih stabala čiji je član. Izvorni čvor je izvor za sva stabla. Dobijen sadržaj parnjak prosleđuje unapred određenim stablima. Sistem se na ovaj način deli u podsisteme u kojima svaki od podsistema predstavlja jedno stablo. Radi bržeg distribuiranja sadržaja svakom podsistemu se dodeljuje jedinstven paket.

Topologija zasnovana na višestrukim stablima rešava neke od problema upotrebe samo jednog stabla, ali uvodi dodatne probleme \cite{DeBoever07}. Prvo, na ovakav način postoji više putanja od izvora do lista pa je potreban značajno složeniji algoritam prosleđivanja. Drugo, kako stabla imaju različite podatke potrebno je obezbediti da svi podaci dođu do svakog parnjaka. Treće, prilikom konstrukcije mreže treba kreirati stabla koja su što je moguće manje dubine čime bi se obezbedio brži protok podataka. Dodatno, javlja se povremena potreba prebacivanja parnjaka iz jednog stabla u drugo uz dodatni zadatak balansiranja dubina svih stabala. Četvrti problem leži u balansiranju kapaciteta stabala. 

\subsection{Topologija mreže}
\label{P2P.3.2}

Sistem zasnovan na mreži se fokusira na kreiranje topologije koja se brzo adaptira na promene. Glavna ideja pri pravljenju ove topologije bila je da parnjaci provode što je moguće manje vremena za održavanje pokrivača i brzi oporavak mreže pri njihovom napuštanju. Prednost ove topologije je brža i jednostavnija konstrukcija pokrivača. Sa druge strane, u ovakvoj konstrukciji potrebno je uložiti značajno vreme za implementaciju algoritma za prosleđivanje sadržaja između parnjaka. Primer topologije mreže nalazi se na slici \ref{fig:topologija-mreze}.
 
\begin{figure}[!ht]
  \centering
  \includegraphics[width=1.05\textwidth]{slike/mesh-topology.jpg}
  \caption{Primer topologije mreže}
  \label{fig:topologija-mreze}
\end{figure}
\par

Značajan kontrast se može primetiti ukoliko se posmatraju razlike između topologije mreže i topologije drveta \ref{P2P.3.1}. Kod topologije drveta kreiranje same topologije je složenije i zahteva više vremena, ali jednom kad se postigne željeni dizajn mreže prosleđivanje podataka je jednostavno. Kod topologije mreže situacija je potpuno drugačija. Sama izrada pokrivača je krajnje jednostavna, jeftina i brza, ali je prosleđivanje podataka značajno kompleksnije. 

U mrežnom pokrivaču glavni cilj parnjaka je da održavaju veliki broj dolaznih konekcija \cite{Shen:2009}. U slučaju izlaska komšije iz mreže, njegov uticaj na parnjake je minimalan. Pokrivač se obično gradi na distribuirani način i svaki parnjak je svestan samo malog dela učesnika, njegovih komšija. Način biranja komšija se razlikuje od politike (eng. \textit{policy}) do politike,
pri čemu je u svim slučajevima ovaj proces brz i jednostavan. Primer brze i jednostavne politike je slučajno biranje komšija. 

U odnosu na topologiju drveta, prosleđivanje podataka u mreži ne odvija se pravolinijski. Nedostatak konkretne strukture koja bi definisala put kojim paket treba ići čini nemogućim određivanje unapred putanje kojom paket treba proći. Zbog navedenog svojstva prosleđivanje paketa je bazirano na lokalnim odlukama koje svaki parnjak donese. Parnjaci donose odluke prema informacijama koje imaju o komšijama. Paket se može razmeniti primenom guranja (eng. \textit{push}) ili povlačenja (eng. \textit{pull}). Kod guranja parnjak određuje koji paket šalje svakom komšiji, a kod povlačenja određuje koji paket traži od svakog komšije. Oba pristupa imaju dobre i loše strane. 

Pristup sa guranjem je efikasniji u mreži sa ograničenim odlaznim kapacitetom jer izbegava višestruke zahteve za pakete. Pristup sa povlačenjem je dobar izbor za mreže sa ograničenim dolaznim kapacitetom jer parnjak može da kontroliše koliko brzo će primati pakete od komšija. Kod obe šeme zbog lokalne koordinacije propagacije paketa dolazi do blage neefikasnosti u prosleđivanju. Kod sistema guranja ova činjenica se ogleda u višestrukim kopijama koje parnjak može da dobije ako mu više komšija šalje isti paket. Kod sistema povlačenja može se desiti da komšije zaguše parnjaka sa previše zahteva za paketima. Takođe sistemi povlačenja unose dodatan sadržaj u mrežu (eng. \textit{overhead}) s obzirom da se paket mora tražiti.

\subsection{Hibridna topologija}
\label{P2P.3.3}

Hibridna topologija kombinuje prednosti topologija drveta i mreže. Preciznije, predstavlja kombinaciju robusnosti mrežne topologije sa jednostavnošću i efikasnošću prosleđivanja paketa koju nudi topologija drveta.

U hibridnoj topologiji, mreža je podeljena na više podmreža slično kao kod topologije drveta \cite{Shen:2009}. Da bi parnjak mogao da prima pakete, mora pronaći roditelje koji će mu prosleđivati pakete iz svih podmreža. Kada se novi parnjak pridružuje mreži, on dobija listu svih parnjaka na koje može da se poveže. Zatim parnjak među njima pokušava da odabere komšije tako da pokrije sve podmreže. Nakon što je pronašao takve parnjake povezuje se na njih. Glavna mana ovakvog sistema je ta što je dobijanje pravilne topologije težak zadatak. Često se može desiti da mala greška dovodi do situacije da parnjak ili grupa parnjaka bude izolovana od ostatka mreže.

\section{Protok video sadržaja}
\label{P2P.4}

Od samog nastanka, video je važan mediji za komunikaciju i zabavu. Sačinjen je od serije slika koje se smenjuju odgovarajućom brzinom dajući privid neprekidnog kretanja. Ovaj trik bio je poznat još u drugom veku u Kini, ali je ostao nepoznat ostatku sveta sve do 19. veka. Otkrićem kamere 1888. godine omogućeno je "automatsko" hvatanje i čuvanje pojedinačnih komponenti slike na filmskoj traci. Emitovanje televizijskog signala, nakon izuma 1928. godine, omogućilo je bilionima ljudi širom sveta uživo praćenje snimljenog sadržaja. Zahvaljujući širokoj dostupnosti signala, primarni vid zabave i dobijanja informacija umesto novina i radija postala je televizija.

Tokom većeg dela 20. veka, jedini način dobijanja televizijskog signala bio je preko vazduha ili kabla. Ranih 2000-tih internet doživljava veliki rast propusnog opsega. Pored rasta propusnog opsega svakodnevno se razvijaju bolji algoritmi kompresije video snimaka čime se omogućava da se sa značajno manje podataka predstavi isti snimak bez velikih gubitaka na kvalitetu.
Takođe, značajnu ulogu u svemu igra i Murov zakon \cite{Moore} činjenicom da eksponencijalni porast snage računara omogućava rešavanje raznovrsnijih problema. Navedena svojstva omogućavaju da sistem za dostavu video sadržaja u realnom vremenu (eng. \textit{streaming}) putem interneta postane moguć. Dostava video sadržaja u realnom vremenu omogućava emitovanje pre preuzimanja celokupnog sadržaja. Video se neprekidno šalje i emitovanje se omogućava odmah po pristizanju ili sa određenom zadrškom \cite{Telecom}.

Internet je dizajniran za slanje paketa bez mogućnosti kontrole toka. Ovakav način transfera ne pogoduje slanju vremenski neprekidno baziranog saobraćaja kao što su video i audio sadržaji \cite{Tanenbaum}. Glavni razlog je što dostava u realnom vremenu ima određeni poredak kojim se mora emitovati kao i vremenska ograničenja. Na primer, video sadržaj se mora emitovati neprekidno slika za slikom. U slučaju da podatak ne stigne na vreme, proces emitovanja će biti prekinut.

Danas se internet sve više koristi za razmenu multimedijalnog sadržaja umesto statičnog teksta i grafika \cite{Beggs:1999}. Aplikacije koje zahtevaju dostavu u realnom vremenu velikom broju korisnika su, između ostalog, Internet TV, prenos sportskih događaja, online igrice i edukacija preko daljine. Usled ovako širokog spektra primena istraživači već skoro trideset godina istražuju odgovarajuću podršku aplikacijama u obliku IP višesmernog emitovanja (eng. \textit{multicast}). IP višesmerno emitovanje je vrsta slanja paketa u kome se dodaje posebna zastavica (eng. \textit{flag}) u IP pakete kojom se signalizira ruterima da je paket potrebno poslati grupi računara paralelno \cite{Deering:1990}. Međutim, usled ozbiljnih problema sa skaliranjem i podrškom na višim nivoima funkcionalnosti, višesmerno emitovanje nije zaživelo šire. Visoki troškovi mreža za isporuku sadržaja (eng. \textit{Content Delivery Networks (CDN)}) i obezbeđivanja odgovarajućeg propusnog opsega su dva glavna faktora koji ograničavaju ovakav vid slanja samo na mali deo izdavača Internet usluga, koji višesmerno emitovanje prevashodno koriste za dostavu kvalitetnog video sadržaja IPTV. Za dostavu sadržaja pretplatnicima izdavači koriste usmerivače paketa (eng. \textit{packet switching}). 

U nadolazećim godinama, postoji značajni interes za korišćenje mreže ravnopravnih računara za dostavu sadržaja u realnom vremenu. Dve ključne stavke čine pomenuti pristup primamljivim. Prvo, ovakva tehnologija ne zahteva podršku mrežne infrastrukture što je čini jeftinom i jednostavnom za izradu. Drugo, u takvoj tehnologiji, svi koji primaju sadržaj isti taj sadržaj dele dalje, što obezbeđuje visok nivo skalabilnosti. Pored velikog interesa i uloženog truda u pronalaženje odgovarajućeg P2P sistema, ova tema je još uvek otvorena. 
Glavna tačka spoticanja P2P mreže za dostavu sadržaja u realnom vremenu je susret sa drugačijim poteškoćama u odnosu na uobičajne probleme mreže. Kod dostave sadržaja u realnom vremenu posebno se mora voditi računa o kašnjenju signala, dok u slučaju preuzimanja sadržaja kašnjenje nije kritično. Zapravo, prihvatljivo je preuzimanje sadržaja u trajanju od nekoliko sati pa čak i dana. Različitost i striktniji zahtevi prenosa video sadržaja u realnom vremenu zahteva fundamentalno drugačiji dizajn i pristup.



\section{Modelovanje mreže}
\label{P2P.5}

U idealnom slučaju, P2P sistem za dostavu sinhronzovanog sadržaja bi trebao postići jednako dobre performanse uzimajući u obzir sve bitne metrike. Pri postizanju ovog cilja neophodno je prevazići različite izazove povezane sa okruženjem u kojem P2P sistem radi i korisničkim ponašanjem. Izazovi su mnogobrojni i različiti pri čemu u nastavku navodimo četiri glavna.

Prvo, P2P sistem treba biti što je moguće otporniji na talasanje. Talasanje može da utiče na kontinualnost signala koji se prikazuje korisniku uzimajući u obzir činjenicu da napuštanje parnjaka može usporiti ili potpuno onemogućiti distribuciju podataka u mreži.
Kada parnjak napusti mrežu, svi parnjaci kojima je on bio dobavljač sadržaja moraju pronaći novog dobavljača u što kraćem roku. 

Drugo, ovakav sistem treba imati dobro skaliranje. Kako mreža raste, povećava se broj parnjaka kao i ukupno vreme potrebno da se sadržaj pošalje svim parnjacima. Za nesmetanu reprodukciju ovakvog sadržaja, kašnjenje reprodukcije takođe treba da raste. Kada se u obzir uzme i skalabilnost, izazov je dvostruki. Dodatno, porast prosečnog kašnjenja reprodukcije sadržaja treba biti takav da nema veliki uticaj na ukupno iskustvo korisnika (eng. \textit{user experience}). Drugo, sistem treba održavati što manje kašnjenje čime se obezbeđuje bolji privid prenosa uživo.

Treći izazov sa kojim se susreću P2P sistemi su različiti propusni opsezi parnjaka. Uspešnost samog sistema zavisi od prosečnog odlaznog kapaciteta parnjaka i jačine enkodovanja video snimka. Danas, propusni opsezi parnjaka mogu biti veoma različiti. Dodatnu poteškoću predstavlja asimetrična priroda pristupa internetu koja dovodi do toga da parnjaci značajno više sadržaja mogu preuzeti nego poslati. Zbog ovih razloga, P2P sistem treba biti izgrađen tako da se heterogenost propusnog opsega koristi efektivno.

Četvrti izazov je da prosečna brzina slanja parnjaka može biti smanjena usled nevoljnosti parnjaka za doprinos mreži. Parnjaci koji žele da prime podatke ali ne i da ih dalje dele sa ostatkom mreže se nazivaju grebatori (eng. \textit{free-riders}). Grebatori su opasnost za sistem jer koriste resurse bez daljeg deljenja pa samim tim smanjuju prosečnu brzinu deljenja u mreži. Stoga, važno je da se P2P sistem dizajnira tako da podstiče doprinos mreži.


%\pangrami

%\pangrami

% ------------------------------------------------------------------------------
\chapter{Zaključak}
% ------------------------------------------------------------------------------


% ------------------------------------------------------------------------------
% Literatura
% ------------------------------------------------------------------------------
\literatura

% ==============================================================================
% Završni deo teze i prilozi
\backmatter
% ==============================================================================

% ------------------------------------------------------------------------------

\end{document}
